% !TEX TS-program = xelatex
% !TEX encoding = UTF-8 Unicode
% !Mode:: "TeX:UTF-8"

\documentclass{resume}
\usepackage{zh_CN-Adobefonts_external} % Simplified Chinese Support using external fonts (./fonts/zh_CN-Adobe/)
%\usepackage{zh_CN-Adobefonts_internal} % Simplified Chinese Support using system fonts
\usepackage{linespacing_fix} % disable extra space before next section
\usepackage{cite}

\begin{document}
\pagenumbering{gobble} % suppress displaying page number

\name{林萌}

\basicInfo{
  \email{lemon\_wonder@outlook.com} \textperiodcentered\
  \phone{(+86) 135-529-87847} \textperiodcentered\
}
\section{\faGraduationCap\  教育背景}
\datedsubsection{\textbf{电子科技大学}}{2013 -- 2016}
\textit{硕士研究生}\ 计算机应用技术
\datedsubsection{\textbf{电子科技大学}}{2009 -- 2013}
\textit{学士}\ 信息安全\

\section{\faUsers\ 项目经历}
\datedsubsection{\textbf{蚂蚁金服}}{2017年7月 -- 至今}
\role{}{OceanBase分布式数据库研发工程师}
OceanBase是面向云时代的金融级分布式数据库,在多租户的架构下,支持了MySQL模式和Oracle模式,同时具有高可用、可扩展、高并发等分布式系统的特点。目前蚂蚁金服所有的核心业务都使用OceanBase数据库,项目地址:https://oceanbase.alipay.com/
\begin{itemize}
  \item 对分布式系统和数据库系统有较好的了解,有多年分布式存储系统的研发经验,掌握数据库系统的基本概念和原理。在OceanBase项目中深入学习了分布式数据库系统的实现,并主要承担了SQL引擎部分的研发工作,对存储和事务模块也有一定的了解,对数据库产品有很好的理解和学习能力。
  \item 了解常见数据库MySQL和Oracle的功能特性,掌握SQL引擎技术实现:词法解析、语义解析、逻辑计划生成、物理计划生成和计划执行。在OceanBase系统中开发了很多MySQL和Oracle的功能特性:回收站功能,主要通过隐藏元数据对用户不可见实现逻辑删除,目前支持库、表进入回收站;数据类型、表达式和字符集相关功能开发;DML功能的维护和功能开发,进行部分DML算子的重构和功能完善,支持更多的分布式特性;其它如flashback query、生成列、伪列等许多功能点的实现。在研发工作中,也学习了MySQL一些功能实现:了解MySQL字符集的源码实现;掌握开源项目go-mysql对MySQL协议的部分实现。
  \item OceanBase采用多版本机制管理元数据,OceanBase中元数据采用表格进行存储,主要分为核心元数据和用户元数据,对元数据的修改主要是DDL语句。在元数据模块开发实现了truncate table和alter table的一些功能,开发和完善核心元数据的管理机制,同时保证兼容性升级。
  \item 修复mysqltest测试环境、线下测试环境和线上环境中的正确性问题和性能问题,具有复现问题、定位问题和解决问题的能力,能够从整体和单条SQL进行性能问题处理。在OceanBase研发工作中修复了很多问题。
\end{itemize}

\datedsubsection{\textbf{美团}}{2016年3月 -- 2017年6月}
\role{}{分布式存储研发工程师}
MSS(美团云对象存储服务)是经过美团内部反复验证的,高可靠、高可用、海量、安全的对象存储服务。兼容S3协议,允许指定对象的持久化级别(多副本或者纠删码);支持光纤互联机房间的异地备份和本地读写优化;保证副本之间强一致;保证在集群掉电或少于设计数量的磁盘损坏的情况下数据不丢失;自动化的扩容与故障隔离;高并发情况下毫秒级别的写入性能以及较高的网络和磁盘带宽;集群高可用性。
\begin{itemize}
  \item 了解对象存储系统的整体架构,掌握美团对象存储系统的设计和实现原理。在使用时展现用户、桶、对象的逻辑存储结构,采用类HayStack的对象管理方式,同时支持各种不同大小的用户对象,并提供EC纠删码功能,进一步降低存储成本。
  \item 参与设计和实现了单机存储引擎store模块的部分功能:黑名单机制,主要处理节点间网络故障和抖动导致的各种问题;实现副本异步加载机制,首次只加载必要副本信息,实现了store节点的快速重启;实现错误隔离机制,主要通过主动发现和被动发现损坏的磁盘和存储副本。
  \item 参与实现store模块的写逻辑:通过多线程并发写入、队列模型、I/O聚合、批量提交等技术减少随机I/O,实现多磁盘并发写入,通过权重机制处理磁盘性能抖动。
  \item 编写元数据处理逻辑:元数据存储在分布式数据库系统中,使用go语言开发元数据的管理逻辑,如租户创建、对象插入等。
  \item 对store模块进行一些性能优化:使用火焰图、日志分析、tracelog等手段对性能问题进行定位。
\end{itemize}

% Reference Test
%\datedsubsection{\textbf{Paper Title\cite{zaharia2012resilient}}}{May. 2015}
%An xxx optimized for xxx\cite{verma2015large}
%\begin{itemize}
%  \item main contribution
%\end{itemize}

\section{\faCogs\ IT 技能}
% increase linespacing [parsep=0.5ex]
\begin{itemize}[parsep=0.5ex]
  \item 多年C/C++编程经验,良好的编程规范,熟练使用gdb、automake、git和vim等工具。
  \item 熟悉linux系统的使用,掌握常用的操作命令和运维命令,可以使用shell脚本和python开发一些运维工具。
  \item 多年分布式系统开发经验,有论文和项目经验积累,熟悉raft算法和paxos算法,熟悉常见的分布式系统的设计。
  \item 热爱开源,喜欢分布式和数据库系统。
\end{itemize}

\section{\faHeartO\ 获奖情况}
\datedline{电子科技大学ACM竞赛三等奖}{2011 年6 月}
\datedline{三星应用开发助跑计划}{2013 年3 月}

\section{\faInfo\ 其他}
% increase linespacing [parsep=0.5ex]
\begin{itemize}[parsep=0.5ex]
  \item GitHub: https://github.com/lemon0910
  \item 性格:热爱开源,参与技术分享和交流,积极乐观,能与人进行良好的合作和沟通。
  \item 语言: 英语 - 熟练(六级)
\end{itemize}

%% Reference
%\newpage
%\bibliographystyle{IEEETran}
%\bibliography{mycite}
\end{document}
