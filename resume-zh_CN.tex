% !TEX TS-program = xelatex
% !TEX encoding = UTF-8 Unicode
% !Mode:: "TeX:UTF-8"

\documentclass{resume}
\usepackage{zh_CN-Adobefonts_external} % Simplified Chinese Support using external fonts (./fonts/zh_CN-Adobe/)
%\usepackage{zh_CN-Adobefonts_internal} % Simplified Chinese Support using system fonts
\usepackage{linespacing_fix} % disable extra space before next section
\usepackage{cite}

\begin{document}
\pagenumbering{gobble} % suppress displaying page number

\name{林萌}

\basicInfo{
  \email{lemon\_wonder@outlook.com} \textperiodcentered\
  \phone{(+86) 135-529-87847} \textperiodcentered\
}
\section{\faGraduationCap\  教育背景}
\datedsubsection{\textbf{电子科技大学}}{2013 -- 2016}
\textit{硕士研究生}\ 计算机应用技术
\datedsubsection{\textbf{电子科技大学}}{2009 -- 2013}
\textit{学士}\ 信息安全\

\section{\faUsers\ 项目经历}
\datedsubsection{\textbf{蚂蚁金服}}{2017年7月 -- 至今}
\role{}{OceanBase分布式数据库内核SQL研发工程师}
OceanBase是蚂蚁完全自主研发的金融级分布式关系数据库,目前蚂蚁金服所有的核心业务都使用OceanBase数据库,项目地址:https://oceanbase.alipay.com/
\begin{itemize}
  \item 了解OceanBase的总体架构设计和OceanBase高可用、可扩展、高并发等特性的实现原理。
  \item 掌握SQL引擎的词法解析、语义解析、逻辑计划生成、物理计划生成和计划执行模块的实现。
  \item 掌握OceanBase元数据管理模块的实现和功能开发。
  \item OceanBase是一款云时代的多租户数据库,实现了MySQL功能兼容和Oracle功能兼容,在工作中对MySQL和Oracle的大部分功能点有很好的了解,并具有快速的数据库功能点学习能力。
  \item 进行功能开发:回收站功能、字符集、flashback query语句、各种DML语句功能点、函数和表达式支持等。
  \item 修复mysqltest测试、线上环境和线下测试环境中的bug和慢SQL,具有复现问题、定位问题和解决问题的能力。
\end{itemize}

\datedsubsection{\textbf{美团}}{2016年3月 -- 2017年6月}
\role{}{分布式存储研发工程师}
MSS(美团云对象存储服务)是经过美团内部反复验证的,高可靠、高可用、海量、安全的对象存储服务。兼容S3协议,允许指定对象的持久化级别(多副本或者纠删码);支持光纤互联机房间的异地备份和本地读写优化;保证副本之间强一致;保证在集群掉电或少于设计数量的磁盘损坏的情况下数据不丢失;自动化的扩容与故障隔离;高并发情况下毫秒级别的写入性能以及较高的网络和磁盘带宽;集群高可用性。
\begin{itemize}
  \item 了解对象存储系统的整体架构。
  \item 参与设计和实现store(存储节点)黑名单机制(处理节点间网络故障和抖动),store异步数据检测校验机制和gc流程。
  \item 开发store模块的读写逻辑:采用线程队列、批量提交等技术提高写入性能,使用强一致的副本策略和冻结技术保证数据的一致性。
  \item 开发store模块的元数据管理:主要包括副本的创建、多副本管理、副本状态的维护等工作。
  \item 对store模块进行一些性能优化:使用火焰图、日志分析、tracelog等手段对性能问题进行定位。
\end{itemize}

% Reference Test
%\datedsubsection{\textbf{Paper Title\cite{zaharia2012resilient}}}{May. 2015}
%An xxx optimized for xxx\cite{verma2015large}
%\begin{itemize}
%  \item main contribution
%\end{itemize}

\section{\faCogs\ IT 技能}
% increase linespacing [parsep=0.5ex]
\begin{itemize}[parsep=0.5ex]
  \item 多年C/C++编程经验,良好的编程规范,熟练使用gdb、automake、git和vim等工具。
  \item 多年linux服务端开发经验,熟悉linux环境编程和系统多线程编程。
  \item 熟悉linux系统的使用,掌握常用的操作命令和运维命令,可以使用shell脚本和python开发一些运维工具。
  \item 多年分布式系统开发经验,有论文和项目经验积累,熟悉raft算法和paxos算法,熟悉常见的分布式系统的设计。
  \item 热爱开源,喜欢分布式和数据库系统。
\end{itemize}

\section{\faHeartO\ 获奖情况}
\datedline{电子科技大学ACM竞赛三等奖}{2011 年6 月}
\datedline{三星应用开发助跑计划}{2013 年3 月}

\section{\faInfo\ 其他}
% increase linespacing [parsep=0.5ex]
\begin{itemize}[parsep=0.5ex]
  \item GitHub: https://github.com/lemon0910
  \item 性格:热爱开源,参与技术分享和交流,积极乐观,能与人进行良好的合作和沟通。
  \item 语言: 英语 - 熟练(六级)
\end{itemize}

%% Reference
%\newpage
%\bibliographystyle{IEEETran}
%\bibliography{mycite}
\end{document}
