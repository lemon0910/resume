% !TEX TS-program = xelatex
% !TEX encoding = UTF-8 Unicode
% !Mode:: "TeX:UTF-8"

\documentclass{resume}
\usepackage{zh_CN-Adobefonts_external} % Simplified Chinese Support using external fonts (./fonts/zh_CN-Adobe/)
%\usepackage{zh_CN-Adobefonts_internal} % Simplified Chinese Support using system fonts
\usepackage{linespacing_fix} % disable extra space before next section
\usepackage{cite}

\begin{document}
\pagenumbering{gobble} % suppress displaying page number

\name{林萌}

\basicInfo{
  \email{lemon\_wonder@outlook.com} \textperiodcentered\
  \phone{(+86) 135-529-87847} \textperiodcentered\
}
\section{\faGraduationCap\  教育背景}
\datedsubsection{\textbf{电子科技大学}}{2013 -- 2016}
\textit{硕士研究生}\ 计算机应用技术
\datedsubsection{\textbf{电子科技大学}}{2009 -- 2013}
\textit{学士}\ 信息安全\

\section{\faUsers\ 项目经历}
\datedsubsection{\textbf{蚂蚁金服}}{2017年7月 -- 至今}
\role{}{OceanBase分布式数据库内核SQL研发工程师}
OceanBase是蚂蚁完全自主研发的金融级分布式关系数据库,目前蚂蚁金服所有的核心业务都使用OceanBase数据库,项目地址:https://oceanbase.alipay.com/
\begin{itemize}
  \item 熟悉了解OceanBase的整体架构设计,了解SQL引擎的整体流程研发和设计,主要工作是SQL模块兼容性功能开发和性能调优。
  \item 利用元数据Schema实现回收站机制。
  \item 研发和维护OceanBase的数据类型框架。
  \item 实现各种特性的DML语句。
  \item 进行安全特性的整体设计和秘钥管理服务的开发。
  \item 参与实现新版本字符集框架开发和gbk字符集功能开发。
  \item 参与双11版本的SQL部分性能优化,满足版本对性能要求。
\end{itemize}

\datedsubsection{\textbf{美团}}{2016年3月 -- 2017年6月}
\role{}{分布式存储研发工程师}
MSS(美团云对象存储服务)是经过美团内部反复验证的,高可靠、高可用、海量、安全的对象存储服务。兼容S3协议,允许指定对象的持久化级别(多副本或者纠删码);支持光纤互联机房间的异地备份和本地读写优化;保证副本之间强一致;保证在集群掉电或少于设计数量的磁盘损坏的情况下数据不丢失;自动化的扩容与故障隔离;高并发情况下毫秒级别的写入性能以及较高的网络和磁盘带宽;集群高可用性。
\begin{itemize}
  \item 了解系统的整体架构。
  \item 参与设计和实现store(存储节点)黑名单机制(处理节点间网络故障和抖动),store异步数据检测校验机制和gc流程。
  \item 开发store模块的核心读写逻辑:采用线程队列、批量提交等技术提高写入性能,使用强一致的副本策略和冻结技术保证数据的一致性。
  \item 开发store模块的元数据管理:主要包括副本的创建、多副本管理、副本状态的维护等工作,保证多线程写入的数据安全。
  \item 对store模块进行一些性能优化:使用火焰图、日志分析、tracelog等手段对性能问题进行定位。
  \item 元数据存储使用分布式数据库系统,编写go数据库客户端相关的业务逻辑,保证元数据更新和查询的正确性。
  \item 解决go客户端高并发情况下的一些死锁问题,使用相关pprof工具对元数据客户端进行问题定位和性能调优。
  \item 编写元数据(存储在分布式数据库中)冷备份模块:主要利用元数据特性进行高效select扫表,使用go的高并发实现数据快速备份。
  \item 编写和维护项目的一些编译安装脚本、多机自动化测试脚本、单机测试脚本等脚本程序,对原有脚本进行一些改进。
\end{itemize}

% Reference Test
%\datedsubsection{\textbf{Paper Title\cite{zaharia2012resilient}}}{May. 2015}
%An xxx optimized for xxx\cite{verma2015large}
%\begin{itemize}
%  \item main contribution
%\end{itemize}

\section{\faCogs\ IT 技能}
% increase linespacing [parsep=0.5ex]
\begin{itemize}[parsep=0.5ex]
  \item 多年C/C++编程经验,良好的编程规范,熟悉gcc,g++,automake等编译链接工具,熟练使用gdb进行程序调试,有大型C/C++项目开发经验。
  \item 熟练使用go编写程序,有go语言项目开发经验、问题诊断和调优经验。
  \item 多年linux服务端开发经验,熟悉linux环境编程和系统多线程编程。
  \item 熟悉linux系统的使用,掌握常用的操作命令和运维命令,可以使用shell脚本和python开发一些运维工具。
  \item 了解分布式系统的基本原理,有一定的论文和项目积累,熟悉raft算法和paxos算法,熟悉常见的分布式系统的设计。
  \item 热爱开源,追求技术,了解一些开源系统的的实现原理,阅读一些开源系统代码。
\end{itemize}

\section{\faHeartO\ 获奖情况}
\datedline{电子科技大学ACM竞赛三等奖}{2011 年6 月}
\datedline{三星应用开发助跑计划}{2013 年3 月}

\section{\faInfo\ 其他}
% increase linespacing [parsep=0.5ex]
\begin{itemize}[parsep=0.5ex]
  \item 技术博客: http://lemon0910.github.io
  \item GitHub: https://github.com/lemon0910
  \item 性格:热爱开源,参与技术分享和交流,积极乐观,能与人进行良好的合作和沟通。
  \item 语言: 英语 - 熟练(六级)
\end{itemize}

%% Reference
%\newpage
%\bibliographystyle{IEEETran}
%\bibliography{mycite}
\end{document}
